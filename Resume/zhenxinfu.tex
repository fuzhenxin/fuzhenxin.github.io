%%%%%%%%%%%%%%%%%%%%%%%%%%%%%%%%%%%%%%%%%
% Medium Length Professional CV
% LaTeX Template
% Version 2.0 (8/5/13)
%
% This template has been downloaded from:
% http://www.LaTeXTemplates.com
%
% Original author:
% Rishi Shah 
%
% Important note:
% This template requires the resume.cls file to be in the same directory as the
% .tex file. The resume.cls file provides the resume style used for structuring the
% document.
%
%%%%%%%%%%%%%%%%%%%%%%%%%%%%%%%%%%%%%%%%%

%----------------------------------------------------------------------------------------
%	PACKAGES AND OTHER DOCUMENT CONFIGURATIONS
%----------------------------------------------------------------------------------------

\documentclass{resume} % Use the custom resume.cls style

%\usepackage{CJKutf8}

\usepackage[UTF8]{ctex}
\usepackage{url}

\usepackage[left=0.7in,top=0.6in,right=0.7in,bottom=0.6in]{geometry} % Document margins
\newcommand{\tab}[1]{\hspace{.2667\textwidth}\rlap{#1}}
\newcommand{\itab}[1]{\hspace{0em}\rlap{#1}}
\name{付振新(Zhenxin Fu)}
\address{15650708568 \\ fuzhenxin95@gmail.com \\ \url{https://zhenxinfu.com}}
\address{Github: \url{http://www.github.com/fuzhenxin}}


\begin{document}


% \begin{rSection}{Education}

% {\bf Wangxuan Institute of Computer Technology (WICT), Peking University} \hfill {\em Sept 2018 - Present} 
% \\ Master in Natural Language Processing
% \\Formerly the institute of computer science and technology.\\
% {\bf School of Electronics Engineering and Computer Science, Peking University} \hfill {\em Sept 2014 - Jul 2018}
% \\ Bachelor of Computer Science
% \\ Department of Computer Science and Technology
% \end{rSection}

% \begin{rSection}{Competition and Internship}
% \textbf{Student Cluster Competition at SC16, SC17, ASC19, \& SC19} \hfill {\em Sept 2016 - Present}\\
% I am in charge of reproducing results of a given paper on our own cluster. We are selected to contribute to a special issue of the journal, Parallel Computing. In ASC19 and SC19, I worked as a coach. \\
% \textbf{Intern at Tencent Group} \hfill {\em Apr 2017 - Aug 2017} \\
% Test Engineer for virtual assistants and name entity recognition task using sequence labeling.\\
% \textbf{Intern at Alibaba Group} \hfill {\em Aug 2018 - Present}\\
% Research Intern for Natural Languge Processing in AliMe Team \& DAMO Academy.
% \end{rSection}

% \begin{rSection}{Publications}

%     [1] \textbf{Style Transfer in Text: Exploration and Evaluation}, AAAI, 2018. \textbf{Citation>100}\\
%     {\it \textbf{Zhenxin Fu}, Xiaoye Tan, Nanyun Peng, Dongyan Zhao, Rui Yan.} \\
%     We propose unsupervised sequence to sequence model to transfer text from style A to style B. For example, converting paper title to news title.

%     [2] \textbf{Query-bag Matching with Mutual Coverage for Information-seeking Conversations in E-commerce}, CIKM (Short), 2019. \\
%     {\it \textbf{Zhenxin Fu}, Feng Ji, Wenpeng Hu, Wei Zhou, Dongyan Zhao, Haiqing Chen, Rui Yan.} \\
%     A bag containing lots of questions corresponds to a same answer. Ranking the bag directly helps the information-seeking conversation.

%     [3] \textbf{Student Cluster Competition 2017, Team Peking University: Reproducing Vectorization of the Tersoff Multi-Body Potential on the Intel Broadwell Architecture}, Parallel Computing (Journel), 2018. \\
%     {\it \textbf{Zhenxin Fu}, Lei Yang, Wenbin Hou, Zhuohan Li, Yifan Wu, Yihua Cheng, Xiaolin Wang, Yun Liang.} \\
%     Report of reproducibility challenge for the student cluster competition at SC17.

%     [4] \textbf{Joint Learning of Word Representation and Sense Representation for Unsupervised Word Sense Disambiguation}, AAAI Student Abstract and Poster Program, 2020. \\
%     {\it Jie Wang, \textbf{Zhenxin Fu}, Moxin Li, Haisong Zhang, Dongyan Zhao, Rui Yan.} \\
%     Unsupervised WSD by learning the sense representation through the supervision from word embedding.

%     [5] \textbf{Semi-supervised Text Style Transfer: Cross Projection in Latent Space}, EMNLP-IJCNLP, 2019. \\
%     {\it Mingyue Shang, Piji Li, \textbf{Zhenxin Fu}, Lidong Bing, Dongyan Zhao, Shuming Shi, Rui Yan.} \\
%     Semi-supervsed text style transfer with cross projection in latent space: cross training autoencoder, style transfer and latent space projection.
    
%     [6] \textbf{Multilingual Dialogue Generation with Shared-Private Memory}, NLPCC (English), 2019. \textbf{Outstanding Paper Award}. \\
%     {\it Chen Chen, Lisong Qiu, \textbf{Zhenxin Fu}, Junfei Li, Rui Yan. } \\
%     Multilingual dailog generation with shared-private memory inspired from multi-task learning.
    
%     [7] \textbf{Automated ICD Coding Based on Word Embedding with Entry Embedding and Attention Mechanism (基于融合条目词嵌入和注意力机制的自动ICD编码)}, NLPCC (Chinese), 2019. \\
%     {\it Hongke Zhang, \textbf{Zhenxin Fu}, Qianping Ren, Hui Xu, Dongyan Zhao, Rui Yan.} \\
%     A classification model for International Classification of Disease (ICD) with item embedding and label attention.

%     [8] \textbf{Find a Reasonable Ending for Stories: Does Logic Relation Help the Story Cloze Test?} AAAI Student Abstract and Poster Program, 2019. \\
%     {\it Mingyue Shang, \textbf{Zhenxin Fu}, Hongzhi Yin, Bo Tang, Dongyan Zhao, Rui Yan.} \\
%     Natural language inference is used to promote story cloze test.
    
%     [9] \textbf{Learning to Converse with Noisy Data: Generation with Calibration}, IJCAI-ECAI, 2018. \\
%     {\it Mingyue Shang, \textbf{Zhenxin Fu}, Nanyun Peng, Yansong Feng, Dongyan Zhao, Rui Yan.}  \\
%     Noise training cases harm trainnig process of generation-based dialog system. So we integrate instance weight into dialog system to overcome the problem.

%     [10] \textbf{One "Ruler" for All Languages: Multi-Lingual Dialogue Evaluation with Adversarial Multi-Task Learning}, IJCAI-ECAI, 2018. \\
%     {\it Xiaowei Tong, \textbf{Zhenxin Fu}, Mingyue Shang, Dongyan Zhao, Rui Yan.} \\
%     We apply multi-task learning in open-domain dialog evaluation.

%     [11] \textbf{ParConnect Reproducibility Report}, Parallel Computing (Journal), 2017.\\
%     {\it Lei Yang, Yilong Li, \textbf{Zhenxin Fu}, Zhuohan Li, Wenbin Hou, Haoze Wu, Xiaolin Wang, Yun Liang.} \\
%     The report of reproducibility challenge for the student cluster competition at SC16.


% \end{rSection}


% \begin{rSection}{Others}
% \textbf{Reviewer}: AAAI-2019; ACL-2019; EMNLP-2019; INLG-2019; AAAI-2020, ACL-2020. \\
% \textbf{Teaching assistant}: Introduction to Natural Language Processing: Foundation, Theory and Application, Spring 2019. \\
% \textbf{Scholarship}: [1] Wang Xuan Scholarship, first prize for the undergraduate internship at ICST, 2018; [2] Special Scholarship (7/32), Peking University, 2019. \\
% \textbf{Github}: A paper list for style transfer in text with star>700. Link: \url{https://github.com/fuzhenxin/Style-Transfer-in-Text}\\
% \textbf{Simple Chrome Extension}: A very simple "Chrome Extension" to automatically log in official websites in PKU (iaaa.pku.edu.cn).

% \end{rSection}




\begin{rSection}{教育经历}

    {\bf 北京大学王选计算机研究所(原北京大学计算机科学技术研究所)} \hfill {\em 2018年9月 - 目前} 
    \\ 硕士:自然语言处理 \\
    {\bf 北京大学信息科学技术学院} \hfill {\em 2014年9月 - 2018年7月}
    \\ 学士:计算机科学与技术
 
\end{rSection}
    

\begin{rSection}{竞赛和实习经历}
    \textbf{大学生超算竞赛(SC16, SC17, ASC19, \& SC19)} \hfill {\em 2016年9月 - 目前}\\
    负责在自建集群上复现一篇高性能计算论文。由于我们的较好表现,受邀在ParallelComputing期刊上展示我们的复现报告。在SC16获团体第六名,在SC17获团体第四名。并带队参加了ASC19和SC19,并在ASC19获一等奖。 \\
    \textbf{腾讯} \hfill {\em 2017年4月 - 2017年8月} \\
    智能助手测试实习生,进行智能助手的测试和命名实体识别工作。\\
    \textbf{阿里巴巴} \hfill {\em 2018年8月 - 目前}\\
    阿里巴巴达摩院阿里小蜜团队研究型实习生。
\end{rSection}

\begin{rSection}{论文}
    
    [1] \textbf{Style Transfer in Text: Exploration and Evaluation}, AAAI, 2018. \textbf{Citation>100}\\
    {\it \textbf{Zhenxin Fu}, Xiaoye Tan, Nanyun Peng, Dongyan Zhao, Rui Yan.}
    
    [2] \textbf{Query-bag Matching with Mutual Coverage for Information-seeking Conversations in E-commerce}, CIKM (Short), 2019. \\
    {\it \textbf{Zhenxin Fu}, Feng Ji, Wenpeng Hu, Wei Zhou, Dongyan Zhao, Haiqing Chen, Rui Yan.}

    [3] \textbf{Student Cluster Competition 2017, Team Peking University: Reproducing Vectorization of the Tersoff Multi-Body Potential on the Intel Broadwell Architecture}, Parallel Computing (Journel), 2018. \\
    {\it \textbf{Zhenxin Fu}, Lei Yang, Wenbin Hou, Zhuohan Li, Yifan Wu, Yihua Cheng, Xiaolin Wang, Yun Liang.}

    [4] \textbf{Joint Learning of Word Representation and Sense Representation for Unsupervised Word Sense Disambiguation}, AAAI Student Abstract and Poster Program, 2020. \\
    {\it Jie Wang, \textbf{Zhenxin Fu}, Moxin Li, Haisong Zhang, Dongyan Zhao, Rui Yan.}

    [5] \textbf{Semi-supervised Text Style Transfer: Cross Projection in Latent Space}, EMNLP-IJCNLP, 2019. \\
    {\it Mingyue Shang, Piji Li, \textbf{Zhenxin Fu}, Lidong Bing, Dongyan Zhao, Shuming Shi, Rui Yan.}

    [6] \textbf{Multilingual Dialogue Generation with Shared-Private Memory}, NLPCC (English), 2019. \textbf{Outstanding Paper Award}. \\
    {\it Chen Chen, Lisong Qiu, \textbf{Zhenxin Fu}, Junfei Li, Rui Yan. }

    [7] \textbf{Automated ICD Coding Based on Word Embedding with Entry Embedding and Attention Mechanism (基于融合条目词嵌入和注意力机制的自动ICD编码)}, NLPCC (Chinese), 2019. \\
    {\it Hongke Zhang, \textbf{Zhenxin Fu}, Qianping Ren, Hui Xu, Dongyan Zhao, Rui Yan.}

    [8] \textbf{Find a Reasonable Ending for Stories: Does Logic Relation Help the Story Cloze Test?} AAAI Student Abstract and Poster Program, 2019. \\
    {\it Mingyue Shang, \textbf{Zhenxin Fu}, Hongzhi Yin, Bo Tang, Dongyan Zhao, Rui Yan.}

    [9] \textbf{Learning to Converse with Noisy Data: Generation with Calibration}, IJCAI-ECAI, 2018. \\
    {\it Mingyue Shang, \textbf{Zhenxin Fu}, Nanyun Peng, Yansong Feng, Dongyan Zhao, Rui Yan.} 

    [10] \textbf{One "Ruler" for All Languages: Multi-Lingual Dialogue Evaluation with Adversarial Multi-Task Learning}, IJCAI-ECAI, 2018. \\
    {\it Xiaowei Tong, \textbf{Zhenxin Fu}, Mingyue Shang, Dongyan Zhao, Rui Yan.}

    [11] \textbf{ParConnect Reproducibility Report}, Parallel Computing (Journal), 2017.\\
    {\it Lei Yang, Yilong Li, \textbf{Zhenxin Fu}, Zhuohan Li, Wenbin Hou, Haoze Wu, Xiaolin Wang, Yun Liang.}
    
\end{rSection}
    
    
\begin{rSection}{其他}
    \textbf{审稿人}: AAAI-2019; ACL-2019; EMNLP-2019; INLG-2019; AAAI-2020, ACL-2020. \\
    \textbf{助教}: 自然语言处理导论:基础,理论与应用, 2019年春季学期. \\
    \textbf{奖学金}: [1] 王选奖学金,一等奖, 2018年; [2] 专项奖学金 (7/32), 北京大学, 2019年. \\
    \textbf{Github}: 文本风格迁移列表。Github中Star数超过700. \\链接: \url{https://github.com/fuzhenxin/Style-Transfer-in-Text}
\end{rSection}



\end{document}
