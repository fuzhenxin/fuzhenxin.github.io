% LaTeX Curriculum Vitae Template
%
% Copyright (C) 2004-2009 Jason Blevins <jrblevin@sdf.lonestar.org>
% http://jblevins.org/projects/cv-template/
%
% You may use use this document as a template to create your own CV
% and you may redistribute the source code freely. No attribution is
% required in any resulting documents. I do ask that you please leave
% this notice and the above URL in the source code if you choose to
% redistribute this file.

\documentclass[letterpaper]{article}

\usepackage{xeCJK}
\setCJKmainfont{STSong}

\usepackage{hyperref}
\usepackage{geometry}

% Comment the following lines to use the default Computer Modern font
% instead of the Palatino font provided by the mathpazo package.
% Remove the 'osf' bit if you don't like the old style figures.
\usepackage[T1]{fontenc}
\usepackage[sc,osf]{mathpazo}

% Set your name here
\def\name{Zhenxin Fu}

% Replace this with a link to your CV if you like, or set it empty
% (as in \def\footerlink{}) to remove the link in the footer:
\def\footerlink{https://zhenxinfu.com/zhenxinfu.pdf}

% The following metadata will show up in the PDF properties
\hypersetup{
  colorlinks = true,
  urlcolor = black,
  pdfauthor = {\name},
  pdfkeywords = {economics, statistics, mathematics},
  pdftitle = {\name: Curriculum Vitae},
  pdfsubject = {Curriculum Vitae},
  pdfpagemode = UseNone
}
\geometry{
  body={6.5in, 8.5in},
  left=1.0in,
  top=1.25in
}

\pagestyle{myheadings}
\markright{\name}
\thispagestyle{empty}

\usepackage{sectsty}
\sectionfont{\rmfamily\mdseries\Large}
\subsectionfont{\rmfamily\mdseries\itshape\large}
\setlength\parindent{0em}
\usepackage{CJKutf8}

\begin{document}
%\begin{CJK*}{UTF8}{Fandol}


{\huge 付振新 (Zhenxin Fu)}


\vspace{0.25in}

\begin{minipage}{0.45\linewidth}
  \begin{large}Peking University\end{large}
  \begin{itemize}
    \itemsep=-0.02in
      \item School of Electronics Engineering and Computer Science (EECS)
      \item Institute of Computer Science and Technology (ICST)
  \end{itemize}
  
\end{minipage}
\begin{minipage}{0.45\linewidth}
  \begin{tabular}{ll}
    Phone: & +86 15650708568 \\
    Email: & fuzhenxin95@gmail.com \\
    Github: & \href{http://www.github.com/fuzhenxin}{\tt https://www.github.com/fuzhenxin} \\
    Homepage: & \href{https://zhenxinfu.com}{\tt https://zhenxinfu.com} \\
  \end{tabular}
\end{minipage}


\section*{Personal}
I am a second year graduate student at EECS and ICST, majoring in Computer Science and Natural Language Processing. I develop research interests in style transfer, dialog system and multi-task learning. I am also a member of the student cluster competition team of Peking University, knowing commonsense on popular developing tools. 


\section*{Education}
\begin{itemize}
  \item \textbf{B.S. Computer Science, EECS, Peking University, 2014-2018} : Major in computer science, developing interest in network, database and parallel computing.
  \item \textbf{M.S. Candidate. Natural Language Processing, ICST, Peking University, 2018-Present} : Major in Natural Language Processing, developing interest in style transfer, open-domain conversation.
\end{itemize}


\section*{Competition \& Internship}
\begin{itemize}
  \item \textbf{Student Cluster Competition at SC16, SC17 \& ASC19, 2016, 2017 \& 2019} : I am in charge of reproducing results of a given paper on our own cluster. We are selected to contribute to a special issue of the journal, Parallel Computing, as one of the top teams from reproducibility challenge. In ASC19, I worked as a coach.
  \item \textbf{Tencent Beijing 2017.4--2017.8} : Test Engineer(Intern) for virtual assistants \& name entity recognition task using sequence labeling.
  \item \textbf{Alibaba Group 2018.8--Present} : Research Intern for Natural Languge Processing in AliMe Team \& DAMO Academy.
\end{itemize}


\section*{Publications}
\begin{itemize}


\item \textbf{Query-bag Matching with Mutual Coverage for Information-seeking Conversations in E-commerce}, CIKM (Short), 2019. \\
{\it \textbf{Zhenxin Fu}, Feng Ji, Wenpeng Hu, Wei Zhou, Dongyan Zhao, Haiqing Chen, Rui Yan. } \\
A bag containing lots of questions corresponds to a same answer. Ranking the bag directly helps the information-seeking conversation.

\item \textbf{Semi-supervised Text Style Transfer: Cross Projection in Latent Space}, EMNLP-IJCNLP, 2019. \\
{\it Mingyue Shang, Piji Li, \textbf{Zhenxin Fu}, Lidong Bing, Dongyan Zhao, Shuming Shi, Rui Yan.} \\
Semi-supervsed text style transfer with cross projection in latent space: cross training autoencoder, style transfer and latent space projection.

\item \textbf{Multilingual Dialogue Generation with Shared-Private Memory}, NLPCC (English), 2019. \\
{\it hen Chen, Lisong Qiu, \textbf{Zhenxin Fu}, Junfei Li, Rui Yan. } \\
Multilingual dailog generation with shared-private memory inspired from multi-task learning.

\item \textbf{Automated ICD Coding Based on Word Embedding with Entry Embedding and Attention Mechanism (基于融合条目词嵌入和注意力机制的自动ICD编码)}, NLPCC (Chinese), 2019. \\
{\it Hongke Zhang, \textbf{Zhenxin Fu}, Qianping Ren, Hui Xu, Dongyan Zhao, Rui Yan.} \\
A classification model for International Classification of Disease (ICD) with item embedding and label attention.

\item \textbf{Find a Reasonable Ending for Stories: Does Logic Relation Help the Story Cloze Test?} AAAI Student Abstract and Poster Program, 2019. \\
{\it Mingyue Shang, \textbf{Zhenxin Fu}, Hongzhi Yin, Bo Tang, Dongyan Zhao, Rui Yan.} \\
Natural language inference is used to promote story cloze test.

\item \textbf{Student Cluster Competition 2017, Team Peking University: Reproducing Vectorization of the Tersoff Multi-Body Potential on the Intel Broadwell Architecture}, Parallel Computing (Journel), 2018. \\
{\it \textbf{Zhenxin Fu}, Lei Yang, Wenbin Hou, Zhuohan Li, Yifan Wu, Yihua Cheng, Xiaolin Wang, Yun Liang.} \\
  The report of reproducibility challenge for the student cluster competition at SC17.

\item \textbf{Learning to Converse with Noisy Data: Generation with Calibration}, IJCAI-ECAI, 2018. \\
  {\it Mingyue Shang, \textbf{Zhenxin Fu}, Nanyun Peng, Yansong Feng, Dongyan Zhao, Rui Yan.}  \\
  Noise training cases harm trainnig process of generation-based dialog system. So we integrate instance weight into dialog system to overcome the problem.

\item \textbf{One "Ruler" for All Languages: Multi-Lingual Dialogue Evaluation with Adversarial Multi-Task Learning}, IJCAI-ECAI, 2018. \\
  {\it Xiaowei Tong, \textbf{Zhenxin Fu}, Mingyue Shang, Dongyan Zhao, Rui Yan.} \\
  We apply multi-task learning in open-domain dialog evaluation.

\item \textbf{Style Transfer in Text: Exploration and Evaluation}, AAAI, 2018. \\
  {\it \textbf{Zhenxin Fu}, Xiaoye Tan, Nanyun Peng, Dongyan Zhao, Rui Yan.} \\
  We propose unsupervised sequence to sequence model to transfer text from style A to style B. For example, converting paper title to news title.
  
\item \textbf{ParConnect Reproducibility Report}, Parallel Computing (Journal), 2017.\\
  {\it Lei Yang, Yilong Li, \textbf{Zhenxin Fu}, Zhuohan Li, Wenbin Hou, Haoze Wu, Xiaolin Wang, Yun Liang.} \\
  The report of reproducibility challenge for the student cluster competition at SC16.
\end{itemize}


\section*{Others}
\begin{itemize}
  \item \textbf{Reviewer}: AAAI-2019, ACL-2019, EMNLP-2019, INLG-2019.
  \item \textbf{Teaching assistant}: Introduction to Natural Language Processing: Foundation, Theory and Application, Spring 2019.
  \item \textbf{Scholarship}: Wang Xuan Scholarship, first prize for the undergraduate internship at ICST.
\end{itemize}


\bigskip


\begin{center}
  \begin{footnotesize}
    Last updated: \today \\
    \href{\footerlink}{\texttt{\footerlink}}
  \end{footnotesize}
\end{center}

\clearpage


{\huge 付振新}


\vspace{0.25in}

\begin{minipage}{0.45\linewidth}
  \begin{large}北京大学\end{large}
  \begin{itemize}
    \itemsep=-0.02in
      \item 信息科学技术学院
      \item 计算机科学与技术研究所
  \end{itemize}
  
\end{minipage}
\begin{minipage}{0.45\linewidth}
  \begin{tabular}{ll}
    手机: & +86 15650708568 \\
    电子邮件: & fuzhenxin95@gmail.com \\
    Github: & \href{https://www.github.com/fuzhenxin}{\tt http://www.github.com/fuzhenxin} \\
    Homepage: & \href{http://zhenxinfu.com}{\tt https://zhenxinfu.com} \\
  \end{tabular}
\end{minipage}


\section*{简介}
北京大学计算机科学技术研究所硕士一年级在读,研究方向为自然语言处理。感兴趣的研究领域包括文本风格迁移,开放域对话系统和多任务学习。同时,作为北大超算队队员,参加过SC16、SC17和ASC19大学生超算比赛。


\section*{教育经历}
\begin{itemize}
  \item \textbf{学士学位} ~北京大学~信息科学技术学院~2014-2018:专业为计算机科学与技术,对网络,数据库,并行计算等方向比较有兴趣.
  \item \textbf{硕士在读} ~北京大学~计算机科学技术研究所~2018-今:专业为自然语言处理,对文本风格迁移,开放域对话系统比较有兴趣.
\end{itemize}


\section*{比赛和实习}
\begin{itemize}
  \item \textbf{大学生超算比赛(SC16, SC17, ASC19)} : 在前两次比赛中负责论文复现部分(在自有集群上复现一篇SC会议论文)。由于该项目上较高的成绩,我们受邀在期刊Parallel Computing上展示我们的复现报告。以教练的身份参加ASC19。
  \item \textbf{腾讯北京 ~ 2017.4--2017.8} : 智能助手测试, 命名实体识别.
  \item \textbf{阿里巴巴 ~ 2018.8--今} : 自然语言处理方向研究型实习生,阿里小蜜\&达摩院团队.
\end{itemize}


\section*{论文}

\begin{itemize}

\item \textbf{Query-bag Matching with Mutual Coverage for Information-seeking Conversations in E-commerce}, CIKM (Short), 2019. \\
{\it \textbf{Zhenxin Fu}, Feng Ji, Wenpeng Hu, Wei Zhou, Dongyan Zhao, Haiqing Chen, Rui Yan. } \\
Bag有一些列问题组成,这些问题对应相同的回复。通过用户问题和Bag的匹配增强对话系统。

\item \textbf{Semi-supervised Text Style Transfer: Cross Projection in Latent Space}, EMNLP-IJCNLP, 2019. \\
{\it Mingyue Shang, Piji Li, \textbf{Zhenxin Fu}, Lidong Bing, Dongyan Zhao, Shuming Shi, Rui Yan.} \\
半监督的文本风格迁移,交叉训练风格迁移,自编码器,隐变量映射。

\item \textbf{Multilingual Dialogue Generation with Shared-Private Memory}, NLPCC (English), 2019. \\
{\it hen Chen, Lisong Qiu, \textbf{Zhenxin Fu}, Junfei Li, Rui Yan. } \\
基于共享-私有Memory的多语言对话模型,受多任务学习启发。

\item \textbf{Automated ICD Coding Based on Word Embedding with Entry Embedding and Attention Mechanism (基于融合条目词嵌入和注意力机制的自动ICD编码)}, NLPCC (Chinese), 2019. \\
{\it Hongke Zhang, \textbf{Zhenxin Fu}, Qianping Ren, Hui Xu, Dongyan Zhao, Rui Yan.} \\
病例自动化编码,利用条目表示和类别注意力的分类模型。

\item \textbf{Find a Reasonable Ending for Stories: Does Logic Relation Help the Story Cloze Test?} AAAI Student Abstract and Poster Program, 2019. \\
{\it Mingyue Shang, \textbf{Zhenxin Fu}, Hongzhi Yin, Bo Tang, Dongyan Zhao, Rui Yan.} \\
利用自然语言推理辅助故事结尾预测任务。

\item \textbf{Student Cluster Competition 2017, Team Peking University: Reproducing Vectorization of the Tersoff Multi-Body Potential on the Intel Broadwell Architecture}, Parallel Computing (Journel), 2018. \\
{\it \textbf{Zhenxin Fu}, Lei Yang, Wenbin Hou, Zhuohan Li, Yifan Wu, Yihua Cheng, Xiaolin Wang, Yun Liang.} \\
为SC17大学生超算比赛的论文复现报告。

\item \textbf{Learning to Converse with Noisy Data: Generation with Calibration}, IJCAI-ECAI, 2018. \\
  {\it Mingyue Shang, \textbf{Zhenxin Fu}, Nanyun Peng, Yansong Feng, Dongyan Zhao, Rui Yan.}  \\
  对话数据含有大量的噪声,数据质量对模型效果有较大影响。本文提出将Instance Weight应用到对话生成,使不同质量的数据样本对模型更新有不同影响。

\item \textbf{One "Ruler" for All Languages: Multi-Lingual Dialogue Evaluation with Adversarial Multi-Task Learning}, IJCAI-ECAI, 2018. \\
  {\it Xiaowei Tong, \textbf{Zhenxin Fu}, Mingyue Shang, Dongyan Zhao, Rui Yan.}  \\
  将Multi-task思想引入到对话评测当中,不同语言之间互相促进。

\item \textbf{Style Transfer in Text: Exploration and Evaluation}, AAAI, 2018. \\
  {\it \textbf{Zhenxin Fu}, Xiaoye Tan, Nanyun Peng, Dongyan Zhao, Rui Yan.} \\
  本文提出一种无监督的序列到序列模型,将一句话从一种风格转换为另一种风格。并且提出了两个评价指标作为对该研究方向的促进。
  
\item \textbf{ParConnect Reproducibility Report}, Parallel Computing (Journal), 2017.\\
  {\it Lei Yang, Yilong Li, \textbf{Zhenxin Fu}, Zhuohan Li, Wenbin Hou, Haoze Wu, Xiaolin Wang, Yun Liang.} \\
  为SC16大学生超算比赛的论文复现报告。
\end{itemize}


\section*{其他}
\begin{itemize}
  \item \textbf{审稿人}: AAAI-2019, ACL-2019, EMNLP-2019, INLG-2019.
  \item \textbf{助教}: 自然语言处理导论:基础,理论与应用,2019年春.
  \item \textbf{奖学金}: 北大计算所王选奖学金(一等).
\end{itemize}

% Footer
%\begin{center}
%  \begin{footnotesize}
%    Last updated: \today \\
%    %\href{\footerlink}{\texttt{\footerlink}}
%  \end{footnotesize}
%\end{center}



%\end{CJK*}
\end{document}
